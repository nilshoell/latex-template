%%%%%%%%%%%%%%%%%%%%%%%%%%%%%%%%%%%%%%%%%
% Version 1.7.1 - SITM (2020-07-20)
%
% Created by Vel (vel@latextemplates.com) and Johannes Böttcher
% Based on https://www.latextemplates.com/template/masters-doctoral-thesis
%
% Modified by Nils Höll & Christian Hauff (@christianhauff)
%
% THIS TEMPLATE IS MOSTLY COMPLIANT WITH THE FORMAL THESIS REQUIREMENTS OF THE
% SITM CHAIR AT THE UNIFERSITY OF DUISBURG-ESSEN
% (Chair of Information Systems and Strategic IT Management, Prof. Ahlemann)
% See https://www.sitm.wiwi.uni-due.de/
%
% Template license:
% CC BY-NC-SA 4.0 (https://creativecommons.org/licenses/by-nc-sa/4.0/)
%%%%%%%%%%%%%%%%%%%%%%%%%%%%%%%%%%%%%%%%%


%----------------------------------------------------------------------------------------
%  BASIC DOCUMENT CONFIGURATIONS
%----------------------------------------------------------------------------------------

\documentclass[
  11pt, % The default document font size, options: 10pt, 11pt, 12pt
  oneside, % One side output by default, comment it out to switch to two side with alternating margins
  american, % AE: american, BE: english ngerman for (new) German -> (New spelling, important for line breaks in a word)
  onehalfspacing, % Single line spacing, alternatives: onehalfspacing or doublespacing
  % draft, % Uncomment to enable draft mode (no pictures, no links, overfull hboxes indicated)
  % nolistspacing, % If the document is onehalfspacing or doublespacing, uncomment this to set spacing in lists to single
  liststotoc, % Uncomment to add the list of figures/tables/etc to the table of contents
  % toctotoc, % Uncomment to add the main table of contents to the table of contents
  % parskip, % Uncomment to add space between paragraphs
  % nohyperref, % Uncomment to not load the hyperref package
  headsepline, % Uncomment to get a line under the header
  chapterinoneline, % Uncomment to place the chapter title next to the number on one line
  % consistentlayout, % Uncomment to change the layout of the declaration, abstract and acknowledgements pages to match the default layout
]{MastersDoctoralThesis} % The class file specifying the document structure

% Change depth of section numbering and table of contents representation
\setcounter{secnumdepth}{2} % Numbers down to \subsection; no \subsubsection
\setcounter{tocdepth}{2} % Printed in toc down to \subsection; no \subsubsection

%----------------------------------------------------------------------------------------
%  MARGIN SETTINGS
%----------------------------------------------------------------------------------------

\geometry{
  paper=a4paper, % Change to letterpaper for US letter
  inner=4cm, % Inner margin
  outer=2.5cm, % Outer margin
  bindingoffset=0cm, % Binding offset
  top=2cm, % Top margin
  bottom=2cm, % Bottom margin
  % showframe, % Uncomment to show how the type block is set on the page
}


%----------------------------------------------------------------------------------------
%  THESIS INFORMATION
%----------------------------------------------------------------------------------------

% Author information
\author{Max Mustermann} % Your name, print it elsewhere with \authorname
\matnr{123456} % Matriculation number, print it elsewhere with \matnumber
\mailaddress{max.mustermann@myserver.com} % eMail address, print elsewhere with \email
\addresses{Sample Street 25\\456965 Sample City} % Your address, print it elsewhere with \addressname

% Thesis information
\thesistitle{Catchy Title} % Your thesis title, print it elsewhere with \ttitle
\degree{Master Thesis} % Your degree name, print it elsewhere with \degreename
\subject{Computer Sciences} % Your subject area, this is not currently used anywhere in the template, print it elsewhere with \subjectname
\keywords{\LaTeX, Master Thesis, Programming} % List of keywords, print it elsewhere with \keywordnames

% Environment information
\supervisor{Dr. Supervisor} % Your supervisor's name, print it elsewhere with \supname
\examiner{Prof. Dr. Examiner} % Your examiner's name, print it elsewhere with \examname
\university{\href{https://www.uni-due.de}{University of Duisburg-Essen}} % Your university's name and URL, this is used in the title page and abstract, print it elsewhere with \univname
\faculty{\href{https://www.wiwi.uni-due.de}{Faculty of Business Administration and Economics}} % Your faculty's name and URL, this is used in the title page and abstract, print it elsewhere with \facname


%----------------------------------------------------------------------------------------
%  PACKAGES
%----------------------------------------------------------------------------------------

% ------ FONTS ------
\usepackage[T1]{fontenc} % Output font encoding
\usepackage[utf8]{inputenc} % Required for inputting international characters
\usepackage{textcomp}
\usepackage{mathptmx} % Similar to TNR

% Print URLs in the same font instead of typewriter font
\usepackage{url}
\urlstyle{same}

% Change chapter/section title sizes and switch to Helvetica for titles
\usepackage{helvet}
\usepackage{titlesec}
\titleformat{\chapter}{\fontsize{16}{20}\bfseries\fontfamily{phv}\selectfont}{\thechapter}{1em}{}
\titlespacing*{\chapter}{0pt}{*2.5}{*1.5}
\titleformat{\section}{\Large\bfseries\fontfamily{phv}\selectfont}{\thesection}{1em}{}
\titleformat{\subsection}{\large\bfseries\fontfamily{phv}\selectfont}{\thesubsection}{1em}{}


% ------ BIBLIOGRAPHY ------
\usepackage[autostyle=true]{csquotes}
% Use the biber backend & APA citation style
\usepackage[backend=biber,style=apa,natbib=true]{biblatex}
\addbibresource{sources.bib} % The filename of the bibliography


% ------ FIGURES ------
\usepackage{graphicx}
\usepackage[font=small,labelfont=bf]{caption} % Required for specifying captions to tables and figures
\usepackage{subcaption} % Required for nested captions
\usepackage{wrapfig} % Required for wrapping text around figures
% \usepackage{forest} % For generating directory trees
% https://tex.stackexchange.com/questions/5073/making-a-simple-directory-tree/270761#270761 Forest examples

% ------ TRANSLATIONS ------
\usepackage{translator}

% Include translations
% ---------- CUSTOM TRANSLATIONS ----------

% Define translations for "Chapter"
\newtranslation[to=German]{chapter}
  {Kapitel}
\newtranslation[to=English]{chapter}
  {chapter}

% Define translations for "Figure"
\newtranslation[to=German]{figure}
  {Abbildung}
\newtranslation[to=English]{figure}
  {figure}

% Define translations for "Appendix"
\newtranslation[to=German]{appendix}
  {Anhang}
\newtranslation[to=English]{appendix}
  {appendix}

% ---------- TITLE PAGE ----------

% Define translations for "Study Programme"
\newtranslation[to=German]{studyProgramme}
  {Studiengang}
\newtranslation[to=English]{studyProgramme}
  {Study Programme}

% Define translations for "Submitted by"
\newtranslation[to=German]{submittedBy}
  {Vorgelegt von}
\newtranslation[to=English]{submittedBy}
  {Submitted by}

% Define translations for "Matriculation Number"
\newtranslation[to=German]{matNum}
  {Matrikelnummer}
\newtranslation[to=English]{matNum}
  {Matriculation Number}

% Define translations for "Initial Examiner"
\newtranslation[to=German]{initEx}
  {Erstprüfer}
\newtranslation[to=English]{initEx}
  {Initial Examiner}

% Define translations for "Secondary Examiner"
\newtranslation[to=German]{secEx}
  {Zweitprüfer}
\newtranslation[to=English]{secEx}
  {Secondary Examiner}

% ---------- CUSTOM TRANSLATIONS ----------

% Use \translate{idToPrint} to print the correct translation
% Define translations for ""
% \newtranslation[to=German]{idToPrint}
%   {German Translation}
% \newtranslation[to=English]{idToPrint}
%   {English Translation}


% ------ ACRONYMS ------
\usepackage[nohyperlinks]{acronym}
% \usepackage[nohyperlinks, printonlyused]{acronym} % Prints only used acronyms in overview


% ------ OTHER ------
\usepackage{shortvrb}
\PassOptionsToPackage{hyphens}{url}\usepackage{hyperref}

\usepackage{float}
\restylefloat{table}

\hbadness=99999 %suppresses most underfull hbox warnings


% ------ LISTINGS ------
\usepackage{listings}

% Define JavaScript language support
\lstdefinelanguage{JavaScript}{
  keywords={break, case, catch, continue, debugger, default, delete, do, else, false, finally, for, function, if, in, instanceof, new, null, return, switch, this, throw, true, try, typeof, var, void, while, with},
  morecomment=[l]{//},
  morecomment=[s]{/*}{*/},
  morestring=[b]',
  morestring=[b]",
  sensitive=true
}

% Load required languages
\lstloadlanguages{PHP, XML, HTML, Java, JavaScript, SQL}

% Define colors for highlighting
\definecolor{lstbg}{gray}{0.95}
\definecolor{lstComment}{RGB}{51, 102, 0}
\definecolor{lstKey}{RGB}{0, 51, 204}
\definecolor{lstStr}{RGB}{162, 43, 43}

% Styling
\lstset{
  showtabs=false,
  showspaces=false,
  %tabsize=4,
  %tab=\rightarrowfill,
  showstringspaces=false,
  basicstyle=\ttfamily,
  columns=fullflexible,
  frame=single,
  %frameround=tttt,
  breaklines=true,
  backgroundcolor=\color{lstbg},
  commentstyle=\color{lstComment},
  stringstyle=\color{lstStr},
  keywordstyle=\color{lstKey},
  %prebreak=\mbox{\textbackslash}%,
  literate=
  {á}{{\'a}}1 {é}{{\'e}}1 {í}{{\'i}}1 {ó}{{\'o}}1 {ú}{{\'u}}1
  {Á}{{\'A}}1 {É}{{\'E}}1 {Í}{{\'I}}1 {Ó}{{\'O}}1 {Ú}{{\'U}}1
  {à}{{\`a}}1 {è}{{\`e}}1 {ì}{{\`i}}1 {ò}{{\`o}}1 {ù}{{\`u}}1
  {À}{{\`A}}1 {È}{{\'E}}1 {Ì}{{\`I}}1 {Ò}{{\`O}}1 {Ù}{{\`U}}1
  {ä}{{\"a}}1 {ë}{{\"e}}1 {ï}{{\"i}}1 {ö}{{\"o}}1 {ü}{{\"u}}1
  {Ä}{{\"A}}1 {Ë}{{\"E}}1 {Ï}{{\"I}}1 {Ö}{{\"O}}1 {Ü}{{\"U}}1
  {â}{{\^a}}1 {ê}{{\^e}}1 {î}{{\^i}}1 {ô}{{\^o}}1 {ß}{{\ss}}1
}


%----------------------------------------------------------------------------------------
%  ADDITIONAL COMMANDS
%----------------------------------------------------------------------------------------

% ---------- CUSTOM COMMANDS ----------

% Option to move href url to footnote for printing
\newcommand{\printurl}[2]{ % Options below are mutually exclusive!
   \href{#1}{#2} % Uncomment for standard href
  % #2\footnote{#1} % Uncomment for print mode, URL in footnote
}

% Better highlighting for inline code
\newcommand{\linecode}[1]{%
  \colorbox{lstbg}{\textcolor{lstStr}{\textbf{\texttt{#1}}}}%
}

% Outputs 'Chapter X' (or translation) as a clickable link
\newcommand{\chapref}[1]{%
  \hyperref[#1]{\translate{chapter} \ref{#1}}%
}

% Outputs 'Figure X' (or translation) as a clickable link
\newcommand{\figref}[1]{%
  \hyperref[#1]{\translate{figure} \ref{#1}}%
}

% Outputs 'Appendix X' (or translation) as a clickable link
\newcommand{\appref}[1]{%
  \hyperref[#1]{\translate{appendix} \ref{#1}}%
}


% ------ HYPERSETUP ------
\AtBeginDocument{
  \hypersetup{pdftitle=\ttitle} % Set the PDF's title to your title
  \hypersetup{pdfauthor=\authorname} % Set the PDF's author to your name
  \hypersetup{hidelinks} % Prints all links black; comment out for default LaTeX behavior
}

\begin{document}

% ------ PAGE NUMBERING ------
\frontmatter % Use roman page numbering style (I, II, III, IV...) for the pre-content pages
\pagenumbering{roman} % Capital roman numbers, use "roman" or comment for lower case (i, ii, iii, iv...)

\pagestyle{plain} % Default to the plain heading style until the thesis style is called for the body content

\sloppy

% Include title page
%----------------------------------------------------------------------------------------
%  TITLE PAGE
%----------------------------------------------------------------------------------------

\begin{titlepage}
    \begin{center}
       \vspace*{.04\textheight}
       {\scshape\huge\univname\par\vspace{0.8cm}} % University name
       \textsc{\large \degreename}\\[0.5cm] % Thesis type
  
       \vfill
  
       On the subject\\
       {\large \bfseries \ttitle\par}\vspace{0.4cm} % Thesis title
  
      \vfill
      \vfill
  
       {Submitted to the \facname \,at the \univname}\vspace{0.8cm} % Faculty name
  
       \vfill
       \vfill
       \vfill
       \vfill
       \vfill
  
       \begin{minipage}[t]{0.2\textwidth}
        \begin{flushleft} \large
         By:
        \end{flushleft}
       \end{minipage}
       \begin{minipage}[t]{0.6\textwidth}
        \begin{flushleft} \large
          \authorname\\
          \addressname\\
          \email\\
          Matriculation number: \matnumber
        \end{flushleft}
       \end{minipage}\\[1cm]
  
       \begin{minipage}[t]{0.2\textwidth}
        \begin{flushleft} \large
         Supervisors:
        \end{flushleft}
       \end{minipage}
       \begin{minipage}[t]{0.6\textwidth}
        \begin{flushleft} \large
          \supname\\
          \examname
        \end{flushleft}
       \end{minipage}\\[1cm]
  
       \vfill
  
       Summer term 2020, 4. Semester of studies
       \vfill
    \end{center}
  \end{titlepage}

\cleardoublepage

% Include abstract page
%----------------------------------------------------------------------------------------
%  ABSTRACT PAGE
%----------------------------------------------------------------------------------------
\MakeShortVerb{\|}

\addchaptertocentry{\abstractname}
\section*{\abstractname}

English abstract goes here.



\section*{Zusammenfassung}

Deutscher Abstract kommt hier hin.

%----------------------------------------------------------------------------------------
%  LIST OF CONTENTS/FIGURES/TABLES PAGES
%----------------------------------------------------------------------------------------
{\hypersetup{linkcolor=black} % Black links in all lists
  \tableofcontents % Prints the main table of contents

  \listoffigures % Prints the list of figures

  \listoftables % Prints the list of tables

  \lstlistoflistings %Prints the list of (code) listings
  \addchaptertocentry{Listings}
}


%----------------------------------------------------------------------------------------
%  ABBREVIATIONS
%----------------------------------------------------------------------------------------
  
\chapter{\abbrevname}
\begin{acronym}[AAAAAA]
  \acro{acl}[ACL]{Access Control List}
  \acro{ajax}[AJAX]{Asynchronous JavaScript and XML}
  \acro{api}[API]{Application Programming Interface}
  \acro{css}[CSS]{Cascading Style Sheets}
  \acro{gui}[GUI]{Graphical User Interface}
  \acro{html}[HTML]{Hypertext Markup Language}
  \acro{json}[JSON]{JavaScript Object Notation}
  \acro{php}[PHP]{PHP Hypertext Preprocessor}
  \acro{sql}[SQL]{Structured Query Language}
  \acro{uid}[UID]{User Identifier}
  \acro{url}[URL]{Uniform Resource Locator}
  \acro{ux}[UX]{User Experience}
  % \acro{}[]{}
\end{acronym}
% Use 
% - \ac{id} for standard behavior (\Ac{id} for first letter capitalized)
% - \acs{id} for acronym
% - \acl{id} for long version
% - \acp{id} for plural (with 's' at the end)
% - \acf{id} for the full version (long name + acronym in parentheses)
% - \newacro{id}[Acronym]{Long Name} to define acronyms outside the abbreviations list; e.g. for shorthands
% See https://ctan.mirror.norbert-ruehl.de/macros/latex/contrib/acronym/acronym.pdf

%----------------------------------------------------------------------------------------
%  THESIS CONTENT - CHAPTERS
%----------------------------------------------------------------------------------------

\mainmatter % Begin numeric (1,2,3...) page numbering

\pagestyle{thesis} % Return the page headers back to the "thesis" style

% Include the chapters of the thesis as separate files from the Chapters folder
% Uncomment the lines as you write the chapters

% Kapitel 1

\chapter{Introduction}
\label{chap:introduction}

\begin{itemize}
  \item Hierin steht die Begründung der Arbeit, insbesondere der Zweck und die zu lösende(n) Aufgabe(n).
  \item Das Kapitel beginnt mit der Beschreibung der Problemlage.
  \item Hiervon wird die zu lösende Aufgabe abgeleitet.
  \item Ausgangslage und Zielsituation sind so zu konkretisieren und mit Fakten zu belegen (zu quantifizieren), dass am Ende entschieden werden kann, ob und wie die gestellten Fragen beantwortet sind.
  \item Kurzübersicht der Inhalte
\end{itemize}

Erstes Vorkommen von \ac{ajax}, sowie kurz darauf nochmal \ac{ajax}
% Kapitel 2

\chapter{Foundations} % Main chapter title

\label{chap:foundations}

\begin{itemize}
  \item Hier gehören alle für das weitere Geschehen erforderlichen Grundlagen hin, aber auch nicht mehr, denn:
  \item Die Thesis ist kein Lehrbuch. Hier muss kein Überblick über weit verbreitete Inhalte und Methoden Ihres Themas gegeben werden.
  \item Beschreiben Sie nur das, was Sie im weiteren Verlauf Ihrer Thesis anwenden.
  \item Der fehlende Bezug zwischen Grundlagen und deren Anwendung in der Thesis ist der häufigste Grund für eine suboptimale Bewertung.
  \item Lassen Sie bitte, wenn möglich, das Wort theoretisch weg, denn es kann auch praktische Grundlagen geben.
\end{itemize}

%----------------------------------------------------------------------------------------
%	Section 1 - Title
%----------------------------------------------------------------------------------------
\section{Section 1}

%-----------------------------------
%	Subsection 1.1 - Subtitle
%-----------------------------------
\subsection{Subsection 1.1}



%-----------------------------------
%	Subsection 1.2 - Subtitle
%-----------------------------------
\subsection{Subsection 1.2}



%----------------------------------------------------------------------------------------
%	Section 2 - Title
%----------------------------------------------------------------------------------------
\section{Section 2}


%-----------------------------------
%	Subsection 2.1 - Subtitle
%-----------------------------------
\subsection{Subsection 2.1}


%-----------------------------------
%	Subsection 2.2 - Subtitle
%-----------------------------------
\subsection{Subsection 2.2}


%----------------------------------------------------------------------------------------
%	Section 3 - Title
%----------------------------------------------------------------------------------------
\section{Section 3}

% Kapitel 3

\chapter{Research Approach}

\label{chap:methods}

\begin{itemize}
  \item In diesem Kapitel beschreiben Sie, was Sie getan haben, um die Ziele und die Ergebnisse der Arbeit zu erreichen. Hierzu gehört auch wie, womit, warum (zu welchem Zweck) von wem sonst noch und in welcher Reihenfolge die beschriebenen Aktivitäten durchgeführt wurden.
\end{itemize}

%----------------------------------------------------------------------------------------
%	Section 1 - Title
%----------------------------------------------------------------------------------------
\section{Section 1}

%-----------------------------------
%	Subsection 1.1 - Subtitle
%-----------------------------------
\subsection{Subsection 1.1}



%-----------------------------------
%	Subsection 1.2 - Subtitle
%-----------------------------------
\subsection{Subsection 1.2}



%----------------------------------------------------------------------------------------
%	Section 2 - Title
%----------------------------------------------------------------------------------------
\section{Section 2}


%-----------------------------------
%	Subsection 2.1 - Subtitle
%-----------------------------------
\subsection{Subsection 2.1}


%-----------------------------------
%	Subsection 2.2 - Subtitle
%-----------------------------------
\subsection{Subsection 2.2}


%----------------------------------------------------------------------------------------
%	Section 3 - Title
%----------------------------------------------------------------------------------------
\section{Section 3}

% Kapitel 4

\chapter{Results} % Main chapter title
\label{chap:results}

\begin{itemize}
  \item Hier finden sich alle Ergebnisse.
  \item Die Ergebnisse sind durch Messungen, Formeln, Diagramme, Tabellen und eine logische, für den Leser nachvollziehbare Argumentation zu belegen.
  \item Kein hierfür erforderlicher Beleg (etwa eine Tabelle oder ein Diagramm) wird in den Anhang "`verbannt"'.
  \item Die Kernergebnisse müssen in Abbildungen, Tabellen und Formeln stecken.
  \item Der Leser muss jedes Ergebnis anhand der gegebenen Zahlen, Daten und Fakten (ZDF ) einfach nachvollziehen können, ohne langwierige eigenen Rechnungen.
  \item Im diesem Kapitel dürfen keine grundlegenden Erläuterungen und Begründungen mehr auftauchen.
  \item Wenn alle Ergebnisse beschrieben sind, steht fest, welche Grundlagen benötigt werden und wie das Vorgehen hierfür zu beschreiben ist, damit der Leser es mit Blick auf die Ergebnisse nachvollziehen kann.
\end{itemize}

%----------------------------------------------------------------------------------------
%	Section 1 - Title
%----------------------------------------------------------------------------------------
\section{Section 1}

%-----------------------------------
%	Subsection 1.1 - Subtitle
%-----------------------------------
\subsection{Subsection 1.1}



%-----------------------------------
%	Subsection 1.2 - Subtitle
%-----------------------------------
\subsection{Subsection 1.2}



%----------------------------------------------------------------------------------------
%	Section 2 - Title
%----------------------------------------------------------------------------------------
\section{Section 2}


%-----------------------------------
%	Subsection 2.1 - Subtitle
%-----------------------------------
\subsection{Subsection 2.1}


%-----------------------------------
%	Subsection 2.2 - Subtitle
%-----------------------------------
\subsection{Subsection 2.2}


%----------------------------------------------------------------------------------------
%	Section 3 - Title
%----------------------------------------------------------------------------------------
\section{Section 3}

% Kapitel 5

\chapter{Discussion} % Main chapter title
\label{chap:discussion}


% Kapitel 6

\chapter{Conclusion} % Main chapter title
\label{chap:conclusion}

\begin{itemize}
  \item Hier wird die gesamte Arbeit auf einer Seite, maximal zwei Seiten zusammengefasst.
  \item Insbesondere werden die zentralen Ergebnisse dargestellt.
  \item Es werden Vorschläge beschrieben zur Lösung der offen gebliebenen Fragen, oder derer, die sich im Laufe der Arbeit ergeben haben.
  \item In diesem Kapitel folgt auch eine kritische Analyse der eigenen Arbeit.
  \item Ein Ausblick auf die Bearbeitung offener Fragen kann gegeben werden.
\end{itemize}



%----------------------------------------------------------------------------------------
%  THESIS CONTENT - APPENDICES
%----------------------------------------------------------------------------------------

\appendix % Cue to tell LaTeX that the following "chapters" are Appendices

% Include the appendices of the thesis as separate files from the Appendices folder
% Uncomment the lines as you write the Appendices

% % Appendix A

\chapter{Formatierung} % Main appendix title

\label{AppendixA} % For referencing this appendix elsewhere, use \ref{AppendixA}

\section{Aufzählungen}

\subsection{Itemize}

\begin{itemize}
  \item Erstes Item
  \item Zweites
  Mit Zeilenumbruch
\end{itemize}

\subsection{Enumerate}

\begin{enumerate}
  \item Punkt mit eins
  \item Und die zwei
\end{enumerate}

\subsection{Description}

\begin{description}
  \item [Punkt 1] Zu Punkt eins lässt sich viel sagen
  \item [1 Label] zum labeln von Dingen
\end{description}

\section{Code}

\begin{lstlisting}[language=XML]
  <role rolename="manager-gui"/>
  <user username="admin" password="MySecretPassword" roles="manager-gui"/>
\end{lstlisting}
% % Appendix A

\chapter{Bilder} % Main appendix title

\label{AppendixB} % For referencing this appendix elsewhere, use \ref{AppendixB}

\begin{figure}[h] % h is the float option, in this case: try to print at this point of the code
    \centering %center the picture
    \includegraphics[width=0.75\columnwidth]{Figures/HS_PF_Logo_Grau}
    \caption[Logo HSPF]{The Logo of Hochschule Pforzheim University\\ Source: \cite{WissArbeiten}} %[short caption (reference in list of figures)]{Long caption}
    \label{fig:DellVC}
\end{figure}

% %AppendixC

\chapter{Quellcode} % Main appendix title

\label{AppendixC}

%\lstinputlisting[caption=index.php,label=index.php,language=PHP]{../Webseite/index.php}



%----------------------------------------------------------------------------------------
%  BIBLIOGRAPHY
%----------------------------------------------------------------------------------------

\printbibliography[heading=bibintoc]

%----------------------------------------------------------------------------------------

\end{document}