% Kapitel 2

\chapter{Grundlagen} % Main chapter title

\label{Chapter2}

\begin{itemize}
  \item Hier gehören alle für das weitere Geschehen erforderlichen Grundlagen hin, aber auch nicht mehr, denn:
  \item Die Thesis ist kein Lehrbuch. Hier muss kein Überblick über weit verbreitete Inhalte und Methoden Ihres Themas gegeben werden.
  \item Beschreiben Sie nur das, was Sie im weiteren Verlauf Ihrer Thesis anwenden.
  \item Der fehlende Bezug zwischen Grundlagen und deren Anwendung in der Thesis ist der häufigste Grund für eine suboptimale Bewertung.
  \item Lassen Sie bitte, wenn möglich, das Wort theoretisch weg, denn es kann auch praktische Grundlagen geben.
\end{itemize}

%----------------------------------------------------------------------------------------
%	Abschnitt 1 - Titel
%----------------------------------------------------------------------------------------
\section{Abschnitt 1}

%-----------------------------------
%	Unterpunkt 1.1 - Untertitel
%-----------------------------------
\subsection{Unterabschnitt 1.1}



%-----------------------------------
%	Unterpunkt 1.2 - Untertitel
%-----------------------------------
\subsection{Unterabschnitt 1.2}



%----------------------------------------------------------------------------------------
%	Abschnitt 2 - Titel
%----------------------------------------------------------------------------------------
\section{Abschnitt 2}


%-----------------------------------
%	Unterpunkt 2.1 - Untertitel
%-----------------------------------
\subsection{Unterabschnitt 2.1}


%-----------------------------------
%	Unterpunkt 2.2 - Untertitel
%-----------------------------------
\subsection{Unterabschnitt 2.2}


%----------------------------------------------------------------------------------------
%	Abschnitt 3 - Titel
%----------------------------------------------------------------------------------------
\section{Abschnitt 3}
